\capitulo{1}{Introducción}

Uno de los principales problemas que se encuentran a la hora de hacer un estudio de mercado es el conjunto de datos sobre el que trabajar. Existen muchas maneras de recolectar esos datos como podrían ser encuestas, sondeos, entrevistas o incluso observaciones directas. Es aquí donde entra el proceso de recolección de datos usado en este proyecto: \english{Web scraping}. Este concepto se puede definir como una técnica de obtención de datos de sitios web de forma automatizada, forma parte de un concepto que lo engloba llamado \english{Data mining} o Minería de datos.

Este proceso conlleva una serie de ventajas sobre la recolección de datos de forma manual:

\begin{itemize}
	\item Una de las principales razones por la que utilizar un \english{web scraper} es la automatización, dado que siempre vas a querer extraer los mismos campos o características de un sitio web, este proceso se convierte rápidamente en repetitivo y aburrido. Un \english{web scraper} es capaz de realizar estas acciones de forma relativamente autónoma y mucho más rápido de lo que sería hacerlo de forma manual.
	\item Otra ventaja importante de utilizar un \english{web scraper} sobre la extracción de datos de forma manual sería el almacenamiento de la información extraída. Aunque se podrían llegar a almacenar de la misma forma los datos extraídos de ambas formas, con un \english{web scraper} tienes la posibilidad de personalizar el modo en que se almacenen los datos, de tal forma que sea una propia extensión del proceso de extracción además de poder guardarlo en varios formatos a la vez sin apenas alargar la duración del proceso.
\end{itemize}

Una vez que se dispone de la información necesaria para iniciar el estudio, es probable que se necesite de un procesamiento de esta información para clasificar o identificar los datos deseados. Existen infinidad de procesamientos de datos posibles, desde procesamiento de textos como \english{Text sentiment analysis}\footnote{Proceso automático de identificación y extracción de información subjetiva de un texto.} o categorización, a procesamiento de imágenes como se ha hecho en este proyecto. En este caso, se ha hecho una clasificación automática de imágenes para más tarde extraer información de éstas dependiendo de la categoría en la que han sido clasificadas.